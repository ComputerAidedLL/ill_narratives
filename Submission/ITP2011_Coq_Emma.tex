\documentclass[runningheads,a4paper]{llncs}
\usepackage{marginote}%A enlever avant envoi

\usepackage{amssymb}
\setcounter{tocdepth}{3}
\usepackage{graphicx}
\usepackage{bnf}

\usepackage{url}
%\urldef{\mailsa}\path|{alfred.hofmann, ursula.barth, ingrid.haas, frank.holzwarth,|
%\urldef{\mailsb}\path|anna.kramer, leonie.kunz, christine.reiss, nicole.sator,|
%\urldef{\mailsc}\path|erika.siebert-cole, peter.strasser, lncs}@springer.com|    
\newcommand{\keywords}[1]{\par\addvspace\baselineskip
\noindent\keywordname\enspace\ignorespaces#1}

\usepackage{listings}
\usepackage{bold-extra}
\usepackage{lstcoq}
\lstset{language=Coq,fontadjust=false,columns=fixed,showspaces=false,showstringspaces=false}
\lstset{basicstyle=\tt\small,commentstyle=\it\ttfamily,basewidth={0.475em},keywordstyle=\bf\ttfamily}
\lstset{captionpos=t,float,belowcaptionskip=\medskipamount}


\begin{document}

\mainmatter  % start of an individual contribution

\title{Structural Analysis of Narratives with the Coq Proof Assistant}

% a short form should be given in case it is too long for the running head
\titlerunning{Structural Analysis of Narratives with the Coq Proof Assistant}
% the name(s) of the author(s) follow(s) next
%
% NB: Chinese authors should write their first names(s) in front of
% their surnames. This ensures that the names appear correctly in
% the running heads and the author index.
%
\author{Anne-Gwenn Bosser\inst{1}%
%\thanks{Please note that the LNCS Editorial assumes that all authors have used
%the western naming convention, with given names preceding surnames. This determines
%the structure of the names in the running heads and the author index.}%
\and Pierre Courtieu\inst{2}\and Julien Forest\inst{2} \and Marc Cavazza\inst{1}}
%
%\authorrunning{Lecture Notes in Computer Science: Authors' Instructions}
% (feature abused for this document to repeat the title also on left hand pages)

% the affiliations are given next; don't give your e-mail address
% unless you accept that it will be published
\institute{University of Teesside, School of Computing, Intelligent Virtual Environments Research Group\\
\url{http://ive.scm.tees.ac.uk/}
\and
Conservatoire National des Arts et M\'{e}tiers, Laboratoire CEDRIC, Equipe CPR\\
\url{http://cedric.cnam.fr/}}

%
% NB: a more complex sample for affiliations and the mapping to the
% corresponding authors can be found in the file "llncs.dem"
% (search for the string "\mainmatter" where a contribution starts).
% "llncs.dem" accompanies the document class "llncs.cls".
%

%\toctitle{Lecture Notes in Computer Science}
%\tocauthor{Authors' Instructions}
\maketitle
\begin{abstract}
This paper proposes a novel application of Interactive Proof Assistants for studying the formal properties of Narratives, building on recent work demonstrating the suitability of Intuitionistic Linear Logic as a conceptual model. 
More specifically, we describe a method for modelling narrative resources and actions, together with constraints on the story endings in the form of an ILL sequent. We describe how well-formed interactive narratives can be interpreted from cut-free proof trees of the sequent obtained using Coq. We finally describe how to reason about interactive narratives at the structural level using Coq: by allowing to prove 2nd order properties on the set of all the proofs generated by a sequent, Coq assists the verification of structural narrative properties traversing all the generated stories.

%This paper proposes a novel application of Interactive Proof Assistants, for studying the formal properties of Narratives. This builds on recent work demonstrating the suitability of Intuitionistic Linear Logic (ILL) as a conceptual model for Interactive Narratives, for it provides a theory for representing the basic properties associated to narrative actions, i.e. causality and competition for resources. This extends the philosophy behind the use of LL for the representation of actions semantics in computational linguistics. Linear Logic could in the long-term support narrative generation on a principled basis, by relying on a representation of core properties rather than ad hoc narrative ontologies, such as those associated to AI approaches to narrative generation. As our approach relies on a correspondance between an ILL proof and an Interactive Narrative, a first step in that direction is to study formal properties of narratives using a direct encoding of ILL within the Coq Proof-Assistant.

%More specifically, we give the example of a subset language of ILL which can be used to model resources of a narrative and narrative actions, together with constraints on the story endings and intermediate states of the narrative in the form of an ILL sequent. From this model, we describe how different interactive narratives can be obtained (by cut-free proof trees of the sequent) and verified with Coq, thus allowing to assist the generation of well-formed interactive stories. We finally describe how Coq allows to reason about interactive narratives at the structural level: to formally prove second order properties, such as properties traversing all the interactive narratives a given sequent specification can generate.
\keywords{Linear Logic, Applications of Theorem Provers, Interactive Narratives}
\end{abstract}


\section{Introduction}
%
Narrative representations have always played a prominent role in AI research and Cognitive Science (see~\cite{Richards09} for a recent review and current trends, and~\cite{AAAI10}). More recently, the emergence of new media has created an opportunity for real-world applications of computational models of narratives~\cite{Laurel93,Murray98}. Research in Interactive Storytelling, whose long-term goal is to create narratives which can respond and adapt in real-time to user's reaction, has thus recently become a sustained interest within the AI community~\cite{AAAI07,AAAI09}.

Recent research in Interactive Storytelling has converged on the use of AI techniques (mostly Planning~\cite{Young99}) to generate consistent plots (seen as a sequence of actions corresponding to the backbone of the narrative), but has paradoxically failed to capitalise on narrative representations. Initial hopes of developing computational narratology on the same bases as computational linguistics using narrative models developed in the field of humanities~\cite{Greimas66,Bremond73} have been dashed by the finding that they were closer to ontologies than to proper logical or computational formalisms~\cite{Cavazza06}.

We have recently described in~\cite{bosser10} how Linear Logic(LL)~\cite{GirardTCS87}, and in particular Intuitionistic Linear Logic~\cite{GirardILL87} can provide a suitable conceptual model for Interactive Narratives. Interactive Narratives are modelled as proofs of a sequent written in Linear Logic which describes initial resources and possible narrative actions. This allows to express naturally key properties for Interactive Narratives while supporting a return to first principles of narrative action representation (causality and competition for resources).

%a affiner/preciser. ce paragraphe doit etre accrocheur pour la communaute.
Building on this work, we propose here a first step towards the automation of the structural analysis of interactive narratives using the Coq Proof-Assistant. We describe how to specify narratives on a structural basis only in the form of an ILL sequent, and a dedicated ILL encoding into Coq, with tactics allowing to obtain proofs of such a sequent. We describe how such proofs are interpreted as well-formed narratives. Our encoding of ILL into Coq supports, as have previous work, the assisted generation of ILL proofs, but also assists the reasoning on the properties of proofs and on all the possible proofs of an ILL sequent, through lemmas and tactics. This allows us to explore second order structural properties traversing all the interactive narratives which can be generated from a description of initial and atomic narrative actions and resources.

%Further work: tactics based on patterns of use.
\section{Related Works}
\subsection{Logical Approaches to Interactive Storytelling}
inclure travaux avec LL.\\
meme approche que pour papier ECAI mais en insistant plus sur usages de LL + model checking via petri nets/colored petri nets.\\
LL tool support for IS: model checking w. petri nets. Montrer la difference de l'approche auto/prouveur avec une approche Proof Assistant. Insister sur le fait que meme si ils semblent inclure tous les connecteurs on voit pas bien comment ils s'en sortent pour l'indecidabilite, et que le fragment de LL utilise est vraiment pas clair du tout.
\subsection{Related applications of Linear Logic}
computational linguistics: a ete utilise pour la representation d'actions. Ici on etend a un context d'action narratives.\\
LL and planning: idem, representation d'actions.\\
Referer a ECAI-LL-Emma comme "evidence" que LL presente une theorie de l'action particulierement adaptee a la narration interactive.
\subsection{Proof Assistants and Automatic Provers support for LL}
Encodings existants (plusieurs naifs, certains sans proof-terms (juste oui/non) ce qui ne convient pas, c'est la preuve qui nous interesse pas le fait que ce soit prouvable seulement.Les proches:Dixon Isabelle, LLP. 

Dixon: usage "proche" car pour de la planif + le seul exemple d'encodage d'ILL avec des tactiques un peu elaborees dans un assistant de preuve. Nous, on a pas de tactiques definies, mais cet exemple montre que c'est possible et qu'il y a tout un champ qui a ete tres peu investigue par la recherche, et qui a un gros potentiel d'automatisation/semi-automatisation. Les approches Prouveur automatique  a la LLProver (qui mouline 3 semaines sans que rien ne sorte...) et Petri nets ne permettent pas en l'etat actuel des connaissances de passer au niveau du fragment ILL qui est suffisament expressif. L'approche Assistant de Preuve semi-automatique est une premiere etape, permettant d'etudier dans le futur des tactiques performantes, sans avoir a restreindre l'expressivite du fragment considere. Ces consideration sont pour des travaux futurs sur une automatisation ou automatisation partielle.

\subsection{Proposed Approach}
Pour notre sujet, l'analyse de proprietes de narration interactives, et de specification de narration interactives, avantage qu'il y a a pouvoir faire des preuves au second ordre grace a une approche proof-assistant, comparee a une approche prouveur automatique: Meme sans automatisation, un assistant de preuve permet d'exprimer des proprietes "arbitraires" au second ordre sur une specification de narrations interactives: de raisonner sur un ensemble d'histoires que peut generer la specification. Ceci permet de verifier ces proprietes (qui sont structurelles, et suivant notre analogie, mappee sur la structure de la preuve). C'est ce qu'on discute dans ce papier.
\section{ILL as a Representational Theory for IS}
%
Our approach is based on a formal specification of interactive narratives resources (including narrative actions), initial conditions, and possible ending states in the form of an ILL sequent. We then interpret a given proof of such a sequent as an interactive narrative. A sequent may have multiple proofs. It may therefore specify multiple interactive narratives sharing the same characteristics. When interpreting the proof as an interactive narrative, we look for a trace of the use of the $\multimap$ left rule. This rule is interpreted as the execution of a narrative action. Other rules have an interpretation reflecting the structure of the interactive narrative, such as an external branching choice in an open-world assumption (for instance, end-user interaction), or a concurrency relationship between different subsets of the narrative. 
%The story event corresponding to the exection of the $A_{1}\oplus A_{2}$ left rule enforces that both sub-proofs describe the consumption of resources in the rest of the context with each formulae $A_{i}$ leads to the production of the right-hand side formula. It does correspond in out interpretation to ensuring that each of these possibilities in the interactive narrative is a well-formed story.
\subsection{Modelling of Interactive narratives specification through an ILL sequent}
% description de la modelisation du sequent
%The choice of linear logic for representing interactive narratives was originally motivated by the need to capture the nature of narrative actions, and in particular their causal properties. Indeed, the description of narrative causality requires being able to express the semantic of narrative actions, which is modelled very precisely thanks to the explicit management of resources allowed by Linear Logic~\cite{Bosser10}. Using Linear Logic for Knowledge Representation for action semantics also allows to elegantly avoid the frame problem presented by other AI formalism such as the situation calculus. 
%
The subset language of ILL we define here allows to describe the initial resources of the narrative, the available narrative actions, and constraints on the possible ending states of the narrative. Key to our interpretation, narrative actions are modelled using $\multimap$ which allows a precise description of their impact on the narrative environment. As we work in an open world assumption, external impact on the narrative (for instance user interaction) is modelled by using the $\oplus$ connector for describing choices between possible narrative actions, and by using $\&$ for describing a choice between two possible ending states.

Such a specification of narratives encompasses the description of the available resources and states of the narratives, the description of the semantic of narrative actions through their impact on the context of execution, and the possible ending states of the narrative. The initial sequent thus takes the form $\Gamma , \Delta \vdash \mathtt{Goal}$, where $\Gamma$ is a multiset representing resources and initial conditions, $\Delta$ is a multiset representing the possible narrative actions, and $Goal$ a formula representing the possible ending state of the narrative. A sequent thus provides the knowledge representation base of a set of interactive narratives. 

While the description of the sequent places us in the whole ILL fragment, the modeling language we use for this paper causes restrictions on the structure of formulae. This will in turn cause certain restrictions on the properties of the generated proofs which we can verify. (TODO Si temps permet?)

Figure~\ref{fig:sequent_grammars} describes the modelling of interactive narratives specifications into an ILL sequent.
%
\subsubsection{Resources of an Interactive Narrative}
Part of the left-hand side of the sequent consists in the description of the available resources and initial states. A resource can be an atom, 1, or composed. The formula $\mathtt{Res_{1}}\&\mathtt{Res_{2}}$, expresses the availability of one of the resources. One only of $\mathtt{Res_{i}}$ will be used and the choice depends on the proof, or on the branches of the proof. This allows us to describe how the initial conditions can adapt to a given unfolding of the story. The formula $\mathtt{Res_{1}} \otimes \mathtt{Res_{2}}$ allows to express the availability of both resources. The formula $!\mathtt{Res}$ allows to express the unbounded availability of the resource $\mathtt{Res}$.
\begin{figure}
\begin{grammar}
      [(colon){::=}]
      [(semicolon)$|$]
      [(comma){}]
      [(period){\\}]
      [(quote){\begin{tt}}{\end{tt}}]
      [(nonterminal){$\langle$}{$\rangle$}]
"Res":1;"atom";"Res",$\&$,"Res";"Res",$\otimes$,"Res";!"Res".
"Act":1;"CRes",$\multimap$,"Context";"Act",$\oplus$,"Act";"Act",$\&$,"Act";!,"Act".
"Goal":1;"atom";"Goal",$\otimes$,"Goal";"Goal",$\oplus$,"Goal";"Goal",$\&$,"Goal".\\
"CRes":1;"atom";"CRes",$\otimes$,"CRes".
"Context":"Res";"Act";"Context",$\otimes$,"Context".
\end{grammar}
%
%Resources in $\Gamma$:\\
%$R::= 1| atom | R_{1}\&R_{2} | R_{1} \otimes R_{2} | !R $\\
%Actions in $\Delta$:\\
%$A::= 1 | Ac \multimap Ap | A_{1} \oplus A_{2} | A\& A | !A$\\
%$Ac ::= 1 | atom | Ac_{1} \& Ac_{2} | Ac_{1} \otimes Ac_{2}$\\
%$Ac ::= 1 | atom | Ac_{1} \otimes Ac_{2}$\\
%$Ap::= R | A | Ap_{1} \otimes Ap_{2}$\\
%Ending state $\phi$:\\
%$E::= 1 | atom | E_{1} \otimes E_{2} | E_{1} \oplus E_{2} | E_{1} \& E_{2}$
\caption{Resources ($\mathtt{Res}\in\Gamma$), Actions ($\mathtt{Act}\in\Delta$) and Goal description for the $\Gamma, \Delta \vdash \mathtt{Goal}$ sequent specifying a narrative. Actions consume resources of the form \texttt{RCons} and produce context formulae of the form \texttt{Context}\label{fig:sequent_grammars}}
\end{figure}
\subsubsection{Narrative Actions Representation}
A simple narrative action is of the form $\mathtt{CRes} \multimap \mathtt{Context}$, where \texttt{Cres} is a finite resource description and \texttt{Context} a multiplicative conjunction of resources and actions. Its semantic is thus precisely defined in terms of how it affects the execution environment: the execution of the narrative action corresponds to the application of the $\multimap$ left rule in the proof, consuming the resources modelled by \texttt{CRes} and introducing in the sequent context the formula \texttt{Context} which models resources and actions made available by this execution.

Narrative actions can be composed for offering two types of choices. A composed action $\mathtt{Act_{1}} \oplus \mathtt{Act_{2}}$ corresponds to a choice made externally in an open-world assumption. This is used for modelling the impact of events external to the narrative in an open-world assumption (for instance, user interaction). When such a formula is decomposed using the $\oplus$ rule, the two sub-proofs, which require proving the sequent with each of the subformula replacing the composed action, are interpreted as the two possible unfolding of the story. The proof ensures that each possible sub-narrative is well-formed. A composed action $\mathtt{Act_{1}} \& \mathtt{Act_{2}}$ corresponds to a choice depending on the proof-search, and might differ depending on the branch of the narrative. If both choices successfully produce a different proof of the sequent, this will be interpreted as two different interactive narratives.
\subsubsection{Narrative ending states}
The resulting state of a narrative is modelled in the subset of ILL composed of $\otimes$ and $\oplus$ and $\&$. 
$\mathtt{Goal_{1}} \otimes \mathtt{Goal_{2}}$ expresses that both $Goal_{i}$ states are accessible at the end of the interactive narrative. $\mathtt{Goal_{1}} \oplus \mathtt{Goal_{2}}$ expresses that either state $\mathtt{Goal_{i}}$ is accessible, and the choice depends on the proof search and might differ depending on the unfolding branch of the narrative.$\mathtt{Goal_{1}} \& \mathtt{Goal_{2}}$ expresses that either state should be accessible, and this choice depends on an event external to the narrative, such as user interaction for instance. Similarly to the situation with composed actions $\mathtt{Act_1}\oplus\mathtt{Act_2}$, the sub-proofs following the application of the $\&$ right rule will allow ensuring that the corresponding narratives are well-formed stories.
%Note: we don't need $\multimap (R)$, $! (R)$ as the sequent language is stable: these connectors can't appear on the right side whatever the sequent appearing in the proof, see after.
\subsubsection{Stability of the representation}
Lemme de Pierre: on a jamais d'implication en sous-formule gauche d'une application. Generaliser.
If the initial sequent is of the form $\Gamma , \Delta \vdash \phi$ as defined on figure~\ref{sequent_grammars} then all the sequents appearing in the proof respect the same construction modulo application of the $\otimes$ rule: at every step of the proof, the left context is always composed of resources, actions, and $\otimes$ of resources and actions, and the left formula is always in the language of $\phi$.

Intuition:\\
Partie droite du sequent: la seule maniere de faire apparaitre une formule qui n'est pas une sous formule de $\phi$ c'est d'utiliser $\multimap(L)$. Or cela introduit a droite des formules de type $Ac$ qui sont dans un sous ensemble du language de $\phi$.\\
Partie gauche du sequent:\\
Regles permettant d'ajouter une formule a gauche qui n'est pas une sous formule d'une formule de gauche: seulement  $\multimap(R)$ et on ne l'a pas puisque pas dans le language de phi.\\
Regles decomposant une formule a gauche:\\
$\multimap(L)$ fait apparaitre soit une ressource dans le language de $R \in \Gamma$, soit une action dans le langage de $A\in\Delta$ soit une formule qui est un $\otimes$ d'actions et de resources, et se decompose donc en ressources et actions.\\
$\oplus(L)$ n'est utilise qu'avec des actions a droite et a gauche donc se decompose en 2 actions\\
$\&(L)$ n'est utilise qu'avec 2 actions, ou 2 resources.\\

Pour simplifier l'expression de la propriete et sa demonstration, on peut unifier un peu plus en enrichissant le language de description dans figure~\ref{sequent_grammars}. On considere tous les sequents $\Gamma \vdash \phi$ ou $F\in\Gamma$ est $F::= R|A| F_{1}\otimes F_{2}$.
\subsection{Interpreting a proof as an Interactive narrative}
Interactive narratives can be extracted from proofs from a trace of the order of execution of linear implication. The produced interactive narrative is interpreted recursively from the proof, using the $\nu$ interpretation function described on figure~\ref{fig:interpretation}.
\begin{figure}
\begin{itemize}
\item [Leaf rule \textbf{l}] $\mathbf{\nu(l) = \emptyset}$
\item [Unary rules \textbf{r(P)}] $\mathbf{\nu(r(P)) = \nu(P)}$
\item $\multimap(L)$ rule applied to $A\multimap B$ with subproofs $P_{left}$ and $P_{right}$:\\
$\mathbf{\nu(\multimap(L)[P_{left};P_{right}](A\multimap B)) = \nu(P_{left}) \succ \nu(A \multimap B) \succ(P_{right})}$\\
where $\nu(A \multimap B)$ is the narrative action modelled by $A \multimap B$, and $\succ$ is the precedence relationship in the narrative.
\item $\oplus (L)$ rule with subproofs $P_{left}$ and $P_{right}$:\\ 
$\mathbf{\nu(\oplus(L)[P_{left};P{right}]) = \nu(P_{left}) \triangledown \nu(P_{right})}$\\ 
Where $\triangledown$ models a choice in an open world assumption. Both sub-narrative are possible, but only one will actually be unfolded depending on this choice.
\item $\& (R)$ rule with subproofs $P_{left}$ and $P_{right}$:\\ 
$\mathbf{\nu(\&(L)[P_{left};P{right}]) = \nu(P_{left}) \triangledown \nu(P_{right})}$\\ 
The interpretation is similar as above.
\item $\otimes (R)$ rule with subproofs $P_{left}$ and $P_{right}$:\\ 
$\mathbf{\nu(\otimes(R)[P_{left};P{right}]) = \nu(P_{left}) \| \nu(P_{right})}$ \\
Where $\|$ is the concurrency relationship in the narrative. Both sub-narrative are independant and can be unfolded concurrently.
\end{itemize}
%$\nu$ is defined iteratively on the last rule of the proof tree:
%\begin{itemize}
%\item Leaf rule $l$ :$\mathbf{\nu(l) = \emptyset}$
%\item Unary rules $r$, and with $P$ the subproof, $\mathbf{\nu(r(P)) = \nu(P)}$
%\item $\multimap(L)$ rule applied to $A\multimap B$ with subproofs $P_{left}$ and $P_{right}$:\\
%$\mathbf{\nu(\multimap(L)[P_{left};P_{right}](A\multimap B)) = \nu(P_{left}) \succ \nu(A \multimap B) \succ(P_{right})}$\\
%where $\nu(A \multimap B)$ is the narrative action modelled by $A \multimap B$, and $\succ$ is the precedence relationship in the narrative.
%\item $\oplus (L)$ rule with subproofs $P_{left}$ and $P_{right}$:\\ 
%$\mathbf{\nu(\oplus(L)[P_{left};P{right}]) = \nu(P_{left}) \triangledown \nu(P_{right})}$\\ 
%Where $\triangledown$ models a choice in an open world assumption. Both sub-narrative are possible, but only one will actually be unfolded depending on this choice.
%\item $\& (R)$ rule with subproofs $P_{left}$ and $P_{right}$:\\ 
%$\mathbf{\nu(\&(L)[P_{left};P{right}]) = \nu(P_{left}) \triangledown \nu(P_{right})}$\\ 
%The interpretation is similar as above.
%\item $\otimes (R)$ rule with subproofs $P_{left}$ and $P_{right}$:\\ 
%$\mathbf{\nu(\otimes(R)[P_{left};P{right}]) = \nu(P_{left}) \| \nu(P_{right})}$ \\
%Where $\|$ is the concurrency relationship in the narrative. Both sub-narrative are independant and can be unfolded concurrently.
%\end{itemize}
\caption{Proof to Narrative Interpretation function: the function is defined recursively on the last rule of the proof, first on leaf, then unary rules, then branching rules.\label{fig:interpretation}}
\end{figure}
\subsection{Examples:} Construction of the sequent, generated interactive narrative.
\section{Using the Coq proof assistant for Story Properties Analysis}
\subsection{ILL encoding into Coq}

A lot of encodings of (different fragments and variants of) linear
logic have been proposed in proof assistants. In~\cite{Power99} for
instance, authors present a shallow embedding of ILL in Coq and
performs some simple generic proofs about Hanoi towers for an
arbitrary number of discs using induction.
In~\cite{Sadrzadeh03modallinear} authors present a shallow embedding
of the sequent calculus of classical modal linear logic and performs
some \emph{epistemic} proofs. In~\cite{Kalvala95mechanizinglinear} an
efficient and easy to use implementation of ILL rules in the Isabelle
framework is presented.

In many aspects our modeling of ILL in Coq is similar to these work.
The main original feature of this development is the focus on
properties of \emph{proofs} themselves, not only on provability of
sequents. As in these previous works we provide some (limited)
automation for proving \emph{closed sequent}\marginote{r\'ef. vers la
  section idoine.}, but we also provide reasoning lemmas and tactics
for reasoning on properties of proofs and even on \emph{all possible
  proofs of a sequent}. \marginote{r\'ef. vers la section idoine.}

Formulae, proofs and corresponding convenient notations are defined as
follows (type \texttt{env} being an modeling of multisets):

\marginote{no Cut}
\begin{lstlisting}
  Inductive formula : Type := 
  | Proposition : Vars.t -> formula
  | Implies : formula -> formula -> formula 
  | Otimes : formula -> formula -> formula 
  | Oplus : formula -> formula -> formula 
  | One : formula 
  | Zero : formula 
  | Bang : formula -> formula
  | And : formula -> formula  -> formula 
  | Top : formula.

  Notation "A -o B" := (Implies A B) : ILL_scope.
  Notation  "A ++ B" := (Oplus A B) : ILL_scope.
  Notation  "A ** B" := (Otimes A B) : ILL_scope.
  Notation "1" := One : ILL_scope.
  Notation "0" := Zero : ILL_scope.
  Notation  "! A" := (Bang A) : ILL_scope.
  Notation  "A & B" := (And A B) : ILL_scope.
  Notation  "\Top" := Top : ILL_scope.
  Notation  "x \in \G" := (mem x G = true) : ILL_scope.
  Notation  "x \ \G" := (remove x G) : ILL_scope.
 
  Inductive ILL_proof : env -> formula -> Prop :=
    Id : forall \G p, \G == {p} -> \G |- p
  | Impl_R : forall \G p q, p::\G |- q -> \G |- p -o q
  | Impl_L : forall \G \D \D' p q r, (p -o q) \in \G -> (\G \ (p -o q)) == \D \U \D'
(*@\hfill@*)->  \D |- p -> q::\D' |- r -> \G |- r
  | Times_R : forall \G \D \D' p q , \G == \D \U \D' -> \D |- p -> \D' |- q -> \G |- p ** q
  | Times_L : forall \G p q r , (p ** q) \in \G -> q :: p :: (\G \ (p ** q)) |- r -> \G |- r
  | One_R : forall \G, \G == \O -> \G |- 1
  | One_L : forall \G p , 1 \in \G -> (\G \ 1) |- p -> \G |- p
  | And_R : forall \G p q , \G |- p -> \G |- q -> \G |- (p & q)
  | And_L_1 : forall \G p q r , (p & q) \in \G ->  p::(\G \ (p&q)) |- r -> \G |- r
  | And_L_2 : forall \G p q r , (p & q) \in \G ->  q::(\G \ (p&q)) |- r -> \G |- r
  | Oplus_L : forall \G p q r ,  (p ++ q) \in \G
(*@\hfill@*)->  p :: (\G \ (p ++ q)) |- r -> q :: (\G \ (p ++ q)) |- r -> \G |- r
  | Oplus_R_1 : forall \G p q , \G |- p -> \G |- p ++ q
  | Oplus_R_2 : forall \G p q , \G |- q -> \G |- p ++ q 
  | T_ : forall \G, \G |- \Top
  | Zero_ : forall \G p , 0 \in \G -> \G |- p
  | Bang_D : forall \G p q , !p \in \G -> p :: (\G \ (!p)) |- q -> \G |- q
  | Bang_C : forall \G p q , !p \in \G -> !p :: \G |- q ->  \G |- q
  | Bang_W : forall \G p q , !p \in \G -> \G \ (!p) |- q ->  \G |- q
    where " x |- y " := (ILL_proof x y) : ILL_scope.
\end{lstlisting}

Notice the use of the form ``$\phi\in\Gamma \to \Gamma\vdash\dots$''
instead of ``$\phi,\Gamma\vdash\dots$''. This formulation avoids
tedious manipulations on the environment to match rules. Simple
tactics allow to apply rules and premisses of the form $\phi\in\Gamma$
are discharged automatically (on closed environments).



%quelques details techniques. Comparer avec encodings existants si necessaire (y'en a peu, mais certains pour Isabelle sont plus aboutis avec tactiques). Le truc est simple, expliquer pourquoi.
%\subsection{Specifying Interactive Narratives}
%methode/cuisine pour ecrire le sequent. Faut que ca paraisse facile, et suive naturellement la description de la theorie representationnelle.
\subsection{Unfolding a Story}
Generation "assistee" d'histoires interactives \emph{well-formed}. Bien dire que si pour l'instant on ne fait que de la verif, il est possible de definir des tactiques efficaces (e.g full auto), meme pour ILL. Ca na pas ete tres recherche mais Dixon a certains resultats preliminaires. Un objectif futur est de travailler sur des tactiques, y compris ad-hoc et correspondant a des usage patterns (voir ce qui a ete fait en computational linguistics avec succes), et qu'il n'est pas completement sans espoir de penser a une tactique "trivial" future (full-automation) meme si c'est un vrai probleme car ILL est indecidable. Des arguments places ici doivent etre retires de l'etat de l'art et vice-versa, voir ce qui rend le mieux.

Preuve en tant qu'objet de l'etude et non pas prouvabilite.

\textbf{Cas d'utilisation 2: une preuve du sequent qui suit l'ordre du roman, preuve assistee meme si pas full-auto}

Exemple de la preuve deroulee, de maniere assistee: une instance d'un sequent assez generatif. On montre comment on a produit la preuve en Coq, et on raconte l'histoire correspondante, avec les choix possibles de l'utilisateur. 

Il faut que ca fasse "methode". Ideal, un truc du genre: l'utilisateur choisit une action narrative a executer, et une tactique Coq deroule la suite, jusqu'a ce qu'il y aie une nouvelle formule a decomposer. Le contexte est coupe en 2 en fait: il y a les formules qui correspondent a des actions narratives ou composition d'actions narratives, et des trucs qui sont "juste" des ressources ou des etats consommes par les actions. L'utilisateur ne travaille que sur les premieres et Coq l'aide pour les "details".

Considerations sur les possibles alternatives par rapport a la structure de la specification, pour lier avec la suite: les "choix internes" a la recherche de preuve, l'ordre d'execution des actions. Montrer qu'on ne peut pas construire d'histoires qui sont mal formees du point de vue des ressources disponibles (e.g. dans notre cas, commencer par aller voir Rodolphe et choisir de s'enfuir).
\subsection{2nd order analysis of interactive narratives specification}
(Decrire le passage assiste par Coq au second ordre: C'est ce qu'on a en plus par rapport a un prouveur)
\\
Inspection des proprietes d'une preuve/histoire -> Inspection des proprietes de l'ensemble des histoires specifiees par un sequent.\\
Inspection n'est pas basee forcement sur des proprietes generiques, on inspecte ce qu'on veut!

\textbf{Cas d'utilisation 3: prouver des proprietes structurelles au second ordre, e.g. sur toutes les histoires que genere le sequent.}
Exemple des 2 proprietes-meta, de type "toute histoire generee par cette specification de ressources, conditions initiales, et actions narratives, verifie telle propriete":\\ preuves sur l'accessibilite d'une fin donnee quelque soit la preuve.
%- preuve qu'une action donnee se produira toujours avant une autre, meme sans rapports de precedence/causalite directement encode dans le sequent. Dans l'exemple, a cause d'une configuration particuliere de competition pour la consommation de ressources narratives provoquee par une "interaction" possible. \textbf{Cela met dans cet exemple en evidence une relation structurelle de la specification, qui emerge de l'analyse, et qui est a priori non evidente: elle n'est pas encodee directement dans les specifications.}
%\textbf{Cas d'utilisation 4: exhiber une instance particuliere de narration interactive, qui verifie une propriete structurelle donnee}
%Avec l'exemple: exhiber une histoire interactive ayant plusieurs fins possibles suivant le "chemin" parcouru par l'utilisateur. ajout, au second ordre, de contraintes sur la recherche de preuve. Permet de resoudre des problemes non traites?
%Lorsque tactiques pour automatiser seront definies, cela permettra de rechercher/generer des histoires verifiant certaines proprietes structurelles: presence de fins alternatives etc.
\section{Conclusion}
On a montre que: un couplage ILL et Proof assistant fourni un outil puissant pour etudier les proprietes structurelles de narrations interactives.\\
On a fourni: une methode d'encodage IS en sequent ILL, une methode pour derouler une narration interactive correspondante en utilisant Coq.\\
Insister:\\
A partir d'un encodage/specification des conditions initiales, actions narratives, et open-world input, a faire a la main selon une methode que nous avons definie ici -> une methode reposant sur des tactiques de Coq pour derouler une histoire.

%TPHOLs99) Working with Linear Logic in Coq
%
\bibliography{ITP2011_Coq_Emma}
\bibliographystyle{plain}

\end{document}



%%% Local Variables: 
%%% mode: latex
%%% TeX-PDF-mode: t
%%% ispell-local-dictionary: "british"
%%% TeX-master: t
%%% End: 

% LocalWords:  sequent Coq multisets